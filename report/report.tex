\documentclass[a4paper,12pt]{article}
\usepackage[noabs]{HaotianReport}
\usepackage{matlab-prettifier} % for matlab code

\title{第二次作业}
\author{刘昊天}
\authorinfo{电博181班, 2018310648}
\runninghead{高等电力网络分析}
\studytime{2018年12月}

\newcommand{\vect}[1]{\boldsymbol{#1}}

\begin{document}
    \maketitle
    %\newpage
    \section{牛顿-拉斐逊方法求解潮流方程}
    \subsection{题目1 算例解读}
    \paragraph{题目描述} 通过case30.m中的算例基本数据,明确各节点类型,分别说明在直角坐标和极坐标情况下的已知量和待求量都有哪些,形式上需要几个等式方程。

    查询MATPOWER手册【TODO】可知,结构体中的bus属性对应的bus\_type区分了节点的类型,其中PQ节点值为1,PV节点值为2,参考节点值为3,孤立节点值为4。因此,可以通过以下程序统计各类型节点数量:

    \begin{lstlisting}[style=Matlab-editor,basicstyle=\mlttfamily]
mpc = case30();
%% Watch case30
n_pq = sum(mpc.bus(:, BUS_TYPE) == 1);
n_pv = sum(mpc.bus(:, BUS_TYPE) == 2);
n_ref = sum(mpc.bus(:, BUS_TYPE) == 3);
    \end{lstlisting}
    得到结果,
    \begin{lstlisting}
Bus Type: PQ: 24, PV: 5, REF: 1
    \end{lstlisting}

    对于一个$n=N-1$节点的系统来说,设其中共有$r$个PV节点,则共有$n-r$个PQ节点。在本题中,$n=29,r=5$。

    在直角坐标的情况下,待求变量共$2n$个,用
    $$x=[\vect{e}^T \quad \vect{f}]^T=[e_1,e_2,...,e_n,f_1,f_2,...,f_n]^T$$ 表示,潮流方程为\cref{eq:pfzj},共$2n$个等式。本题中,共58个待求变量(包括29个电压实部与29个电压虚部),58个等式。
    \begin{equation}
      \label{eq:pfzj}
      \begin{cases}
        \Delta P_i = P^{SP}_i-(e_a a_i+ f_i b_i) = 0 \quad i=1,2,...,n\\
        \Delta Q_i = Q^{SP}_i - (f_i a_i - e_i b_i) = 0 \quad i = 1,2,...,n-r\\
        \Delta V_i^2 = (V_i^{SP})^2-(e_i^2+f_i^2)=0 \quad i=n-r+1,...,n
      \end{cases}
    \end{equation}

    在直角坐标的情况下,待求变量共$2n-r$个,用
    $$x=[\vect{\theta}^T \quad \vect{V}]^T=[\theta_1,\theta_2,...,\theta_n,V_1,V_2,...,V_{n-r}]^T$$ 表示,潮流方程为\cref{eq:pfjzb},共$2n-r$个等式,其中$N_i$表示与节点i直接相连的节点编号集合。本题中,共53个待求变量(包括29个相角与24个电压幅值),53个等式。
    \begin{equation}
      \label{eq:pfjzb}
      \begin{cases}
        \Delta P_i = P^{SP}_i - V_i \sum_{j\in N_i}V_j(G_{ij}\cos \theta_{ij}+B_{ij}\sin \theta_{ij}) = 0 \quad i=1,2,...,n\\
        \Delta Q_i = Q^{SP}_i - V_i \sum_{j\in N_i}V_j(G_{ij}\sin \theta_{ij}-B_{ij}\cos \theta_{ij}) = 0 \quad i = 1,2,...,n-r\\
      \end{cases}
    \end{equation}

    \subsection{题目2 程序解释}
    \paragraph{题目描述} 仔细阅读runpf函数及matpower手册,解释matpower在求解潮流方程时各步骤的原理。尝试使用N-R方法(newtonpf)进行潮流方程的求解。注意其对节点类型、边界条件的处理方法,分析并探讨N-R方法算法流程、雅克比矩阵、PQ失配量的处理方法,并学习其中的编程技巧。

    经过对runpf函数的研读,发现MATPOWER在计算电力系统潮流时考虑了各种情况,虽然程序结构比较复杂,但总体上可以分为潮流方程构造与潮流方程求解两部分,并进行了一定的迭代。下面从这两方面依次说明,由于本题的重点在于潮流方程的求解,因此相关部分将重点说明。

    \subsubsection{潮流方程构造}
    \begin{enumerate}
      \item 处理算法的配置。runpf中考虑的配置包括:直流/交流潮流、无功越线情况处理、记录及保存计算结果等。
      \item 计算各节点注入复功率,方法为$S^i=S_{gen}^i-S_{load}^i$。其中,负荷模型可以选择ZIP模型,因此注入功率将成为一个关于电压的函数$S^i(V)$。
      \begin{lstlisting}[style=Matlab-editor,basicstyle=\mlttfamily]
while (repeat) % 用于后续无功越线情况迭代
      %% 匿名函数,表征节点注入功率与电压的关系
      Sbus = @(Vm)makeSbus(baseMVA, bus, gen, mpopt, Vm);
      %% 调用NR法求解潮流方程
      [V, success, iterations] = newtonpf(Ybus, Sbus, V0, ref, pv, pq, mpopt);
      %% 将潮流计算结果更新到模型中
      [bus, gen, branch] = pfsoln(baseMVA, bus, gen, branch, Ybus, Yf, Yt, V, ref, pv, pq, mpopt);
    \end{lstlisting}
      \item 无功越线情况处理。如果开启了无功越线,则runpf程序将对潮流方程多次求解。对于某一次求解结果,如果有PV节点的无功输出超出其限制范围,则将该节点转换为PQ节点(其中无功输出为无功限值),再继续下一次求解,直到没有无功越线的情况出现。根据设置,在出现多个越线节点时,可以选择其中越线最大的节点进行处理,也可以全部进行处理。
      若判断开启无功越线情况处理,则执行以下代码,根据当次潮流计算结果,处理越线节点;否则不再迭代。
    \begin{lstlisting}[style=Matlab-editor,basicstyle=\mlttfamily]
%% 提取所有越线节点,并按照越上限、越下限分别存入mx、mn中。
% 代码略
if ~isempty(mx) || ~isempty(mn)  %% 如果存在越线情况
    %% 检查系统的收敛性,代码略
    if length(infeas) == length(remaining) && all(infeas == remaining) && ...
            (isempty(mx) || isempty(mn))
        %% 若所有剩余的PV节点与REF节点都同时越上限或下限,则该系统无解
        success = 0;
        break;
    end
    %% 如果设置每次只处理一个节点
    if qlim == 2    %% 则选择越线幅度最大的节点,代码略
    end
    %% 保存相应越线限制,并将mn并入mx中,代码略
    %% convert to PQ bus
    %% 关闭发动机并在节点上引入等效的PQ负荷
    gen(mx, QG) = fixedQg(mx);      %% set Qg to binding limit
    gen(mx, GEN_STATUS) = 0;        %% temporarily turn off gen,
    for i = 1:length(mx)            %% (one at a time, since
        bi = gen(mx(i), GEN_BUS);   %%  they may be at same bus)
        bus(bi, [PD,QD]) = ...      %% adjust load accordingly,
            bus(bi, [PD,QD]) - gen(mx(i), [PG,QG]);
    end
    if length(ref) > 1 && any(bus(gen(mx, GEN_BUS), BUS_TYPE) == REF)
        %% 无法处理多松弛节点系统的松弛节点无功越线问题,代码略
    end
    bus(gen(mx, GEN_BUS), BUS_TYPE) = PQ;   %% 调整节点类型
    %% 更新各个类型节点列表
    ref_temp = ref;
    [ref, pv, pq] = bustypes(bus, gen);
    if ref ~= ref_temp
        bus(ref, BUS_TYPE) = REF;
        bus( pv, BUS_TYPE) = PV;
        %% 设置新的参考(松弛)节点
    end
    %% 记录限制过的节点
    limited = [limited; mx];
else
    repeat = 0; %% 停止迭代
end
  end
      \end{lstlisting}
    \end{enumerate}
    \subsubsection{潮流方程求解}
    \begin{enumerate}
      \item 本程序采用了极坐标形式列写潮流方程,如\cref{eq:pfjzb}所示,变量为电压幅值与相角。
      \item 潮流计算的初值为模型给定的初值,由于输入是复电压形式,直接取幅值及相角即可。
      \begin{lstlisting}[style=Matlab-editor,basicstyle=\mlttfamily]
Va = angle(V);
Vm = abs(V);
      \end{lstlisting}
      \item 迭代格式。牛拉法的迭代核心思想为在当前点附近的一阶泰勒展开,确定下降方向。迭代格式如\cref{eq:nr}所示。
\begin{equation}
  \label{eq:nr}
  \begin{cases}
    \Delta \vect{x}^{(k)} = -\vect{J}(\vect{x}^{(k)})^{-1}\vect{f}(\vect{x}^{(k)})\\
    \vect{x}^{(k+1)}=\vect{x}^{(k)}+\Delta \vect{x}^{(k)}
  \end{cases}
\end{equation}
      其中$\vect{J}=\frac{\partial \vect{f}}{\partial \vect{x}^T}$为雅克比矩阵。
      \item 误差矩阵计算。改写\cref{eq:pfjzb}的潮流方程为
      \begin{equation}
        f(x) = \begin{pmatrix}
          \vect{P}^{SP}-\vect{P}(\vect{\theta},\vect{V})\\
          \vect{Q}^{SP}-\vect{Q}(\vect{\theta},\vect{V})
      \end{pmatrix} = 0
      \end{equation}
      而通过网络方程可知,
      \begin{equation}
        \begin{aligned}
          \vect{\dot{I}} &= \vect{Y\dot{V}}\\
          \vect{S}&=diag(\vect{\dot{V}})\overline{\dot{I}}=\vect{\dot{V}}\overline{\vect{Y\dot{V}}}=\vect{P}+j\vect{Q}
        \end{aligned}
      \end{equation}
      因此通过以下代码构造误差矩阵,
      \begin{lstlisting}[style=Matlab-editor,basicstyle=\mlttfamily]
mis = V .* conj(Ybus * V) - Sbus(Vm);
F = [   real(mis([pv; pq]));
        imag(mis(pq))   ];
      \end{lstlisting}
      \item 雅克比矩阵计算。为了计算雅克比矩阵,首先应计算复功率关于电压幅值、相角的偏导数,如 \cref{eq:partialv,eq:partialtheta} 所示。
      \begin{equation}
        \label{eq:partialv}
        \begin{aligned}
          \frac{\partial \vect{\dot{V}}}{\partial \vect{V}^T}&=diag(\left[\frac{\dot{V}_i}{V_i}\right])\\
          \frac{\partial \vect{\dot{I}}}{\partial \vect{V}^T}&=Y\cdot diag(\left[\frac{\dot{V}_i}{V_i}\right])\\
          \frac{\partial \vect{S}}{\partial \vect{V}^T} &=diag(\vect{\dot{V}})\overline{\frac{\partial \vect{\dot{I}}}{\partial \vect{V}^T}} + diag(\overline{\dot{I}})\frac{\partial \vect{\dot{V}}}{\partial \vect{V}^T}
        \end{aligned}
      \end{equation}
      \begin{equation}
        \label{eq:partialtheta}
        \begin{aligned}
          \frac{\partial \vect{\dot{V}}}{\partial \vect{\theta}^T}&=j\cdot diag(\vect{\dot{V}})\\
          \frac{\partial \vect{\dot{I}}}{\partial \vect{\theta}^T}&=jY\cdot diag(\vect{\dot{V}})\\
          \frac{\partial \vect{S}}{\partial \vect{\theta}^T} &=diag(\vect{\dot{V}})\overline{\frac{\partial \vect{\dot{I}}}{\partial \vect{\theta}^T}} + diag(\overline{\dot{I}})\frac{\partial \vect{\dot{V}}}{\partial \vect{\theta}^T}
        \end{aligned}
      \end{equation}

      求出复功率对电压幅值、相角的偏导后,容易分解出有功、无功对电压幅值、相角的偏导,如 \cref{eq:pqpartial} 所示。
      \begin{equation}
        \label{eq:pqpartial}
        \begin{aligned}
          \frac{\partial \vect{S}}{\partial \vect{V}^T} &= \frac{\partial \vect{P}}{\partial \vect{V}^T} + j \frac{\partial \vect{Q}}{\partial \vect{V}^T}\\
          \frac{\partial \vect{S}}{\partial \vect{\theta}^T} &= \frac{\partial \vect{P}}{\partial \vect{\theta}^T} + j \frac{\partial \vect{Q}}{\partial \vect{\theta}^T}\\
        \end{aligned}
      \end{equation}

      重排以上偏导数,即可得到雅克比矩阵,如 \cref{eq:jacobi} 所示。
      \begin{equation}
        \label{eq:jacobi}
        J=\begin{pmatrix}
          \frac{\partial \vect{P}}{\partial \vect{\theta}^T} & \frac{\partial \vect{P}}{\partial \vect{V}^T}\\
          \frac{\partial \vect{Q}}{\partial \vect{\theta}^T} & \frac{\partial \vect{Q}}{\partial \vect{V}^T}
      \end{pmatrix}
      \end{equation}

      如果负荷采用ZIP模型,则也会对雅克比矩阵产生影响。以上步骤体现在程序中,即为以下代码。
      \begin{lstlisting}[style=Matlab-editor,basicstyle=\mlttfamily]
%% 计算偏导数
Ibus = Ybus * V;
diagV       = diag(V);
diagIbus    = diag(Ibus);
diagVnorm   = diag(V./abs(V));
dSbus_dVm = diagV * conj(Ybus * diagVnorm) + conj(diagIbus) * diagVnorm;
dSbus_dVa = 1j * diagV * conj(diagIbus - Ybus * diagV);
%% 计算负荷模型的偏导数
[dummy, neg_dSd_dVm] = Sbus(Vm);
%% 修正偏导数
dSbus_dVm = dSbus_dVm - neg_dSd_dVm;
%% 组成雅克比矩阵
j11 = real(dSbus_dVa([pv; pq], [pv; pq]));
j12 = real(dSbus_dVm([pv; pq], pq));
j21 = imag(dSbus_dVa(pq, [pv; pq]));
j22 = imag(dSbus_dVm(pq, pq));
J = [   j11 j12;
        j21 j22;    ];
      \end{lstlisting}
      \item 收敛条件。
      使用收敛条件为
      \begin{equation}
        \parallel f(x) \parallel_\infty < \varepsilon
      \end{equation}
      其中$\varepsilon$为一给定的容差值。程序中体现为
      \begin{lstlisting}[style=Matlab-editor,basicstyle=\mlttfamily]
normF = norm(F, inf);
if normF < tol
    converged = 1;
end
      \end{lstlisting}
    \end{enumerate}

    \section{分布因子与矫正控制}
    \subsection{题目1 分布因子计算}
    \paragraph{题目描述} 试用课本上学习的方法计算发电机输出功率转移分布因子,并与makePTDF函数的计算结果进行比对。

    根据教材【TODO】,分布因子的计算公式为
    \begin{equation}
      \label{eq:fbyz}
      \begin{aligned}
        &G_{k-i}=\frac{X_{mi}-X_{ni}}{x_k}\\
        &\vect{G}= diag(\left[\frac{1}{x_k}\right])\cdot \vect{A}^T \cdot \vect{B}^{-1}
      \end{aligned}
    \end{equation}
    其中$\vect{X}$矩阵为直流潮流中的$\vect{B}$矩阵的逆,表达见\cref{eq:zlcl},$x_k$为第k条支路的电抗。
    \begin{equation}
      \label{eq:zlcl}
      \begin{aligned}
        &\begin{cases}
          \vect{B}_{ii}&=\sum_{j\in N_i}\frac{1}{x_{ij}}\\
          \vect{B}_{ij}&=-\frac{1}{x_{ij}}
        \end{cases}\\
        &\vect{B}=\vect{A}\cdot diag(\left[\frac{1}{x_k}\right]) \cdot \vect{A}^T
      \end{aligned}
    \end{equation}
    其中$N_i$为节点i的邻居节点集合,$x_{ij}$为节点ij之间的电抗,$\vect{A}$为节点支路关联矩阵。此处$\vect{A},\vect{B}$均不包括平衡节点。

    采用本方法的计算代码见附录,如\cref{lst:expPTDF}所示。以下代码测试了本方法与makePTDF的差距:
    \begin{lstlisting}[style=Matlab-editor,basicstyle=\mlttfamily]
%% PTDF
mp_ptdf = makePTDF(res);
exp_ptdf = expPTDF(res);
fprintf('Calc error: %s\n', norm(mp_ptdf-exp_ptdf));
    \end{lstlisting}
    结果输出为:
    \begin{lstlisting}[style=Matlab-editor,basicstyle=\mlttfamily]
Calc error: 1.523481e-14
    \end{lstlisting}
    表明结果正确。

    \subsection{题目2 边界条件变化估算}
    \paragraph{题目描述} 利用分布因子方法估算潮流方程边界条件发生变化时的各线路潮流分布情况,并与在调整后边界条件下的潮流计算结果进行对比,探讨其中的差别与产生的原因。请根据以下场景进行计算:
    \subsubsection{场景1} 当位于2号节点的发电机有功出力减小0.1、1与10时。

    设调整量为$p=-0.1,-1,-10$,则调整后线路潮流变化为
    \begin{equation}
      \Delta P_k = G_{k-2}p
    \end{equation}
    由于分布因子的计算是基于直流潮流模型的,线路首端与尾端的有功功率一致。而对于交流潮流而言,我们定义线路潮流为首尾潮流的均值,
    \begin{equation}
      P_k = \frac{1}{2}(P_k^F-P_k^T)
    \end{equation}
    其中$P_k^F$为线路首端注入有功,$P_k^T$为线路尾端注入有功,注意由于方向的定义,此处均值中为减号。

    根据此定义,撰写程序计算潮流改变量,并与直接调整发电机功率后潮流计算的结果进行对比,程序见附录,如\cref{lst:q1s1}所示,并采用主程序调用,结果如下。其中误差采用无穷范数进行衡量,定义为\cref{eq:error}。
    \begin{equation}
      \label{eq:error}
      Err_p = \parallel\frac{P_k + G_{k-2}p - P_k'}{P_k'}\parallel_\infty \times 100\%
    \end{equation}
    \begin{lstlisting}[style=Matlab-editor,basicstyle=\mlttfamily]
-0.1 Error: 0.00%
-1 Error: 0.01%
-10 Error: 0.12%
    \end{lstlisting}

    可以看到,随着调整量的增加,采用分布因子的计算方法相对误差不断增大。这是由于分布因子方法采用了直流潮流模型,忽略了系统网损,也忽略了系统的非线性。当调整量较小时,局部线性化的精度较高,因此该方法的估计准确度高;而当调整量较大时,非线性体现明显,局部线性化误差较大,因此该方法估计准确度降低。同时,即便在最大调整量10时,相对误差实际也并不大,这表明在2号节点处,当前状态下的潮流方程对2号节点有功功率的曲率不大,线性程度高。

    为了观察误差的分布情况,我们定义相对调整误差为\cref{eq:error1},该指标可以反映不同支路上调整误差的分布,针对本场景的三种情况,对\cref{eq:error1}作图如\cref{fig:q1s1}所示。可见,调整误差分布基本一致,这也说明系统线性化程度较高。
    \begin{equation}
      \label{eq:error1}
      Err_p' = \parallel\frac{P_k + G_{k-2}p - P_k'}{P_k'p}\parallel_\infty \times 100\%
    \end{equation}
    \begin{figure}[htbp]
      \includegraphics[width=0.95\linewidth]{../meta/q1s1}
      \caption{场景1相对调整误差分布}
      \label{fig:q1s1}
    \end{figure}
    \subsubsection{场景2} 当位于2号节点的发电机有功出力减小0.1、1与10且位于13号节点的发电机有功出力相应增加0.1、1与10以使系统总调整量为0时。

    采用同样的方法,将分布因子计算结果与潮流计算结果对比,得到偏差如下。计算相对调整误差,并绘制分布,如\cref{fig:q1s2}所示。
    \begin{lstlisting}[style=Matlab-editor,basicstyle=\mlttfamily]
-0.1 Error: 0.11%
-1 Error: 1.09%
-10 Error: 19.83%
    \end{lstlisting}

    \begin{figure}[htbp]
      \includegraphics[width=0.95\linewidth]{../meta/q1s2}
      \caption{场景2相对调整误差分布}
      \label{fig:q1s2}
    \end{figure}

    可见,随着调整量的增加,误差不断增大,这是与场景1类似的,原因也是修正量过大使线性化逐渐失效。不同的是,在\cref{fig:q1s2}中,相对调整误差分布随着调整量的增大逐渐变化。特别地,在调整10的情况下,相对误差已经接近20\%。这是由于系统非线性程度高,在不同的调整量下体现出不同的偏差特性。从物理上来讲,当发电机功率大幅调整,会引起网络潮流分布的变化,线路网损也有所增减,因而采用直流潮流进行估计误差较大。

    对比场景1可见,在节点13处使用分布因子的误差,要远大于节点2处。这意味着潮流方程在节点13处的曲率要远大于节点2处的曲率。此案例告诉我们,使用分布因子的精确程度与具体的节点有关。

    \subsection{题目3 矫正控制策略}
    \paragraph{题目描述} 利用分布因子方法计算为达到所期望的控制效果所需的矫正控制策略,并与在调整后边界条件下的潮流计算结果进行对比,探讨两者的差别与产生的原因。请根据以下场景进行计算:
    \subsubsection{场景1} 只调节位于27号节点的发电机的有功出力,其他机组有功出力不变,以使线路27-28上的有功功率降低0.1。
    \subsubsection{场景2} 请设计一种方案,选取一对节点并同时调节节点上发电机对的有功出力,以使线路27-28上的有功功率降低0.1。

    \bibliography{report}
    \bibliographystyle{ieeetr}
    \appendix
    \section{程序清单}
      \lstinputlisting[style=Matlab-editor,basicstyle=\mlttfamily,label=lst:hwtask,caption={全部作业任务}]{../hwtask.m}
      \lstinputlisting[style=Matlab-editor,basicstyle=\mlttfamily,label=lst:expPTDF,caption={发电机有功功率转移分布因子计算}]{../expPTDF.m}
      \lstinputlisting[style=Matlab-editor,basicstyle=\mlttfamily,label=lst:q1s1,caption={分布因子与矫正控制题目2场景1}]{../q1s1.m}
      \lstinputlisting[style=Matlab-editor,basicstyle=\mlttfamily,label=lst:q1s2,caption={分布因子与矫正控制题目2场景2}]{../q1s2.m}
    \label{applastpage}
\iffalse
\begin{itemize}[noitemsep,topsep=0pt]
%no white space
\end{itemize}
\begin{enumerate}[label=\Roman{*}.,noitemsep,topsep=0pt]
%use upper case roman
\end{enumerate}
\begin{multicols}{2}
%two columns
\end{multicols}
\begin{lstlisting}[style=Matlab-editor,basicstyle=\mlttfamily]

\end{lstlisting}
\fi
\end{document}
